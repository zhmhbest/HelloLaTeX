% 【导盲区】
\documentclass[UTF8]{ctexart}
% ctexart ctexrep ctexbook ctexbeamer
% article report book beamer letter

%% 引入宏包

% 未使用ctex*类时需要引入此句支持中文
% \usepackage[UTF8, heading = true]{ctex}

% 用于导入 eps、pdf、jpg、png、bmp
\usepackage{graphicx}
\graphicspath{{figures/}, {pics/}, {imgs/}}

%% 设置CTex
% \ctexset{
%
% }

%% 自定义命令
\newcommand{\degree}{^\circ}

%% 文档信息
\title{\heiti 这是标题}
\author{\kaishu 作者}
\date{\today}

% 【正文区】
\begin{document}
    \maketitle

    \section{Font}

    \subsection{Font Family}

    Default、
    \textrm{Roman Family}、{\rmfamily Roman Family}、
    \textsf{Sans Serif Family}、{\sffamily Sans Serif Family}、
    \texttt{Typewriter Family}、{\ttfamily Typewriter Family}、
    {\songti 宋体}、
    {\fangsong 仿宋}、
    {\heiti 黑体}、
    {\kaishu 楷体}

    \subsection{Font Shape}

    Default、
    \textmd{Medium}、{\mdseries Medium}、
    \textbf{Boldface}、{\bfseries Boldface}、
    \textup{Upright}、{\upshape Upright}、
    \textit{Italic}、{\itshape Italic}、
    \textsl{Slanted}、{\slshape Slanted}、
    \textsc{SmallCaps}、{\scshape SmallCaps}

    \subsection{Font Size}
    
    \subsubsection{General}

    {\tiny 丫}
    {\scriptsize 丫}
    {\footnotesize 丫}
    {\small 丫}
    {\normalsize 丫} % 由文档类参数控制
    {\large 丫}
    {\Large 丫}
    {\LARGE 丫}
    {\huge 丫}
    {\Huge 丫}

    \subsubsection{Ctex}

    {\zihao{0} 丫}
    {\zihao{-0} 丫}
    {\zihao{1} 丫}
    {\zihao{-1} 丫}
    {\zihao{2} 丫}
    {\zihao{-2} 丫}
    {\zihao{3} 丫}
    {\zihao{-3} 丫}
    {\zihao{4} 丫}
    {\zihao{-4} 丫}
    {\zihao{5} 丫}
    {\zihao{-5} 丫}

    \section{公式}

    \subsection{inline}

    左侧内容$f(x)=Ax^2+Bx+C$右侧内容

    \subsection{block}

    左侧内容$$E=mc^2$$右侧内容

    \subsection{equation}

    左侧内容\begin{equation}
        AB^2=BC^2+AC^2
    \end{equation}右侧内容

    \subsection{自定义命令}

    三角形的内角和是$180\degree$

    \section{段落}

    激荡10年,看富豪的变迁实际是在看一部中国的变革史,从地产到互联网,富豪的行业在变,国家经济增长的引擎也在变,未来10年,哪个行业会吸引最多的财富,我已迫不及待,拿起笔准备记录这一切!

    激荡10年,看富豪的变迁实际是在看一部中国的变革史,\\(仅换行不缩进)从地产到互联网,富豪的行业在变,国家经济增长的引擎也在变,未来10年,哪个行业会吸引最多的财富,我已迫不及待,拿起笔准备记录这一切!

    激荡10年,看富豪的变迁实际是在看一部中国的变革史,\par (换行且缩进)从地产到互联网,富豪的行业在变,国家经济增长的引擎也在变,未来10年,哪个行业会吸引最多的财富,我已迫不及待,拿起笔准备记录这一切!

    \section{符号}

    \subsection{空白}

    无空格:()
    \par 空格1/6em:(\thinspace)
    \par 空格1/2em:(\enspace)
    \par 空格1em:(\quad)
    \par 空格2em:(\qquad)
    \par 空格:(\ )
    \par 硬空格:(~)
    \par 自定义 kern -1:({\kern -1em})
    \par 自定义 kern +1:({\kern 1em})
    \par 自定义 hskip -1:({\hskip -1em})
    \par 自定义 hskip +1:({\hskip 1em})
    \par 自定义 hspace -1:(\hspace{-1em})
    \par 自定义 hspace +1:(\hspace{1em})
    \par 占位符 hphantom abc: (\hphantom{abc})
    \par 弹性长度(撑满整个空间): (\hfill)

    \subsection{控制符}

    \$、\#、\%、\&、\{、\}、\_

    \textbackslash
    
    \^{}、\~{}

    \subsection{排版符}

    \S、\P、\dag、\ddag、\copyright、\pounds

    \subsection{标志符}

    \TeX{}、LaTeX{}、LaTeXe{}

    \subsection{引号}

    `abc'

    ``abc''

    \subsection{连字符}

    -、
    --、
    ---

    \subsection{非英文字符}

    \oe \OE
    
    \ae \AE
    
    \aa \AA

    \o \O
    
    \l \L

    \ss \SS

    !` ?`

    \subsection{重音}

    \`o \'o  \^o \~o \=o \.o \''o

    \section{插图}
    
    \includegraphics[scale=0.3]{beauty.jpg}

    \includegraphics[width=0.2\textwidth]{beauty.jpg}

    \includegraphics[height=0.2\textwidth]{beauty.jpg}

    \includegraphics[scale=0.2,angle=-90]{beauty.jpg}

    \section{表格}

    demo1

    \begin{tabular}{l c c c r}
            姓名 & 数学 & 语文 & 英语 & 备注
        \\  张三 & 45 & 58 & 66 & 请等待补考通知
        \\  李四 & 56 & 78 & 44 & 成绩一般
        \\  王五 & 99 & 98 & 95 & 成绩优异
        \end{tabular}
    
    demo2

    \begin{tabular}{l|c c c |r}
        \hline
        姓名 & 数学 & 语文 & 英语 & 备注
    \\  张三 & 45 & 58 & 66 & 请等待补考通知
    \\  李四 & 56 & 78 & 44 & 成绩一般
    \\  王五 & 99 & 98 & 95 & 成绩优异
    \\  \hline
    \end{tabular}

    demo3

    \begin{tabular}{l||c c c||r}
        \hline \hline
        姓名 & 数学 & 语文 & 英语 & 备注
    \\  张三 & 45 & 58 & 66 & 请等待补考通知
    \\  李四 & 56 & 78 & 44 & 成绩一般
    \\  王五 & 99 & 98 & 95 & 成绩优异
    \\  \hline \hline
    \end{tabular}

    demo4

    \begin{tabular}{l|c|c|c|p{2cm}}
        姓名 & 数学 & 语文 & 英语 & 备注
    \\  张三 & 45 & 58 & 66 & 请等待补考通知
    \\  李四 & 56 & 78 & 44 & 成绩一般
    \\  王五 & 99 & 98 & 95 & 成绩优异
    \end{tabular}
\end{document}
